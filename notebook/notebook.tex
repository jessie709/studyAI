
% Default to the notebook output style

    


% Inherit from the specified cell style.




    
\documentclass[11pt]{article}

    
    
    \usepackage[T1]{fontenc}
    % Nicer default font (+ math font) than Computer Modern for most use cases
    \usepackage{mathpazo}

    % Basic figure setup, for now with no caption control since it's done
    % automatically by Pandoc (which extracts ![](path) syntax from Markdown).
    \usepackage{graphicx}
    % We will generate all images so they have a width \maxwidth. This means
    % that they will get their normal width if they fit onto the page, but
    % are scaled down if they would overflow the margins.
    \makeatletter
    \def\maxwidth{\ifdim\Gin@nat@width>\linewidth\linewidth
    \else\Gin@nat@width\fi}
    \makeatother
    \let\Oldincludegraphics\includegraphics
    % Set max figure width to be 80% of text width, for now hardcoded.
    \renewcommand{\includegraphics}[1]{\Oldincludegraphics[width=.8\maxwidth]{#1}}
    % Ensure that by default, figures have no caption (until we provide a
    % proper Figure object with a Caption API and a way to capture that
    % in the conversion process - todo).
    \usepackage{caption}
    \DeclareCaptionLabelFormat{nolabel}{}
    \captionsetup{labelformat=nolabel}

    \usepackage{adjustbox} % Used to constrain images to a maximum size 
    \usepackage{xcolor} % Allow colors to be defined
    \usepackage{enumerate} % Needed for markdown enumerations to work
    \usepackage{geometry} % Used to adjust the document margins
    \usepackage{amsmath} % Equations
    \usepackage{amssymb} % Equations
    \usepackage{textcomp} % defines textquotesingle
    % Hack from http://tex.stackexchange.com/a/47451/13684:
    \AtBeginDocument{%
        \def\PYZsq{\textquotesingle}% Upright quotes in Pygmentized code
    }
    \usepackage{upquote} % Upright quotes for verbatim code
    \usepackage{eurosym} % defines \euro
    \usepackage[mathletters]{ucs} % Extended unicode (utf-8) support
    \usepackage[utf8x]{inputenc} % Allow utf-8 characters in the tex document
    \usepackage{fancyvrb} % verbatim replacement that allows latex
    \usepackage{grffile} % extends the file name processing of package graphics 
                         % to support a larger range 
    % The hyperref package gives us a pdf with properly built
    % internal navigation ('pdf bookmarks' for the table of contents,
    % internal cross-reference links, web links for URLs, etc.)
    \usepackage{hyperref}
    \usepackage{longtable} % longtable support required by pandoc >1.10
    \usepackage{booktabs}  % table support for pandoc > 1.12.2
    \usepackage[inline]{enumitem} % IRkernel/repr support (it uses the enumerate* environment)
    \usepackage[normalem]{ulem} % ulem is needed to support strikethroughs (\sout)
                                % normalem makes italics be italics, not underlines
    

    
    
    % Colors for the hyperref package
    \definecolor{urlcolor}{rgb}{0,.145,.698}
    \definecolor{linkcolor}{rgb}{.71,0.21,0.01}
    \definecolor{citecolor}{rgb}{.12,.54,.11}

    % ANSI colors
    \definecolor{ansi-black}{HTML}{3E424D}
    \definecolor{ansi-black-intense}{HTML}{282C36}
    \definecolor{ansi-red}{HTML}{E75C58}
    \definecolor{ansi-red-intense}{HTML}{B22B31}
    \definecolor{ansi-green}{HTML}{00A250}
    \definecolor{ansi-green-intense}{HTML}{007427}
    \definecolor{ansi-yellow}{HTML}{DDB62B}
    \definecolor{ansi-yellow-intense}{HTML}{B27D12}
    \definecolor{ansi-blue}{HTML}{208FFB}
    \definecolor{ansi-blue-intense}{HTML}{0065CA}
    \definecolor{ansi-magenta}{HTML}{D160C4}
    \definecolor{ansi-magenta-intense}{HTML}{A03196}
    \definecolor{ansi-cyan}{HTML}{60C6C8}
    \definecolor{ansi-cyan-intense}{HTML}{258F8F}
    \definecolor{ansi-white}{HTML}{C5C1B4}
    \definecolor{ansi-white-intense}{HTML}{A1A6B2}

    % commands and environments needed by pandoc snippets
    % extracted from the output of `pandoc -s`
    \providecommand{\tightlist}{%
      \setlength{\itemsep}{0pt}\setlength{\parskip}{0pt}}
    \DefineVerbatimEnvironment{Highlighting}{Verbatim}{commandchars=\\\{\}}
    % Add ',fontsize=\small' for more characters per line
    \newenvironment{Shaded}{}{}
    \newcommand{\KeywordTok}[1]{\textcolor[rgb]{0.00,0.44,0.13}{\textbf{{#1}}}}
    \newcommand{\DataTypeTok}[1]{\textcolor[rgb]{0.56,0.13,0.00}{{#1}}}
    \newcommand{\DecValTok}[1]{\textcolor[rgb]{0.25,0.63,0.44}{{#1}}}
    \newcommand{\BaseNTok}[1]{\textcolor[rgb]{0.25,0.63,0.44}{{#1}}}
    \newcommand{\FloatTok}[1]{\textcolor[rgb]{0.25,0.63,0.44}{{#1}}}
    \newcommand{\CharTok}[1]{\textcolor[rgb]{0.25,0.44,0.63}{{#1}}}
    \newcommand{\StringTok}[1]{\textcolor[rgb]{0.25,0.44,0.63}{{#1}}}
    \newcommand{\CommentTok}[1]{\textcolor[rgb]{0.38,0.63,0.69}{\textit{{#1}}}}
    \newcommand{\OtherTok}[1]{\textcolor[rgb]{0.00,0.44,0.13}{{#1}}}
    \newcommand{\AlertTok}[1]{\textcolor[rgb]{1.00,0.00,0.00}{\textbf{{#1}}}}
    \newcommand{\FunctionTok}[1]{\textcolor[rgb]{0.02,0.16,0.49}{{#1}}}
    \newcommand{\RegionMarkerTok}[1]{{#1}}
    \newcommand{\ErrorTok}[1]{\textcolor[rgb]{1.00,0.00,0.00}{\textbf{{#1}}}}
    \newcommand{\NormalTok}[1]{{#1}}
    
    % Additional commands for more recent versions of Pandoc
    \newcommand{\ConstantTok}[1]{\textcolor[rgb]{0.53,0.00,0.00}{{#1}}}
    \newcommand{\SpecialCharTok}[1]{\textcolor[rgb]{0.25,0.44,0.63}{{#1}}}
    \newcommand{\VerbatimStringTok}[1]{\textcolor[rgb]{0.25,0.44,0.63}{{#1}}}
    \newcommand{\SpecialStringTok}[1]{\textcolor[rgb]{0.73,0.40,0.53}{{#1}}}
    \newcommand{\ImportTok}[1]{{#1}}
    \newcommand{\DocumentationTok}[1]{\textcolor[rgb]{0.73,0.13,0.13}{\textit{{#1}}}}
    \newcommand{\AnnotationTok}[1]{\textcolor[rgb]{0.38,0.63,0.69}{\textbf{\textit{{#1}}}}}
    \newcommand{\CommentVarTok}[1]{\textcolor[rgb]{0.38,0.63,0.69}{\textbf{\textit{{#1}}}}}
    \newcommand{\VariableTok}[1]{\textcolor[rgb]{0.10,0.09,0.49}{{#1}}}
    \newcommand{\ControlFlowTok}[1]{\textcolor[rgb]{0.00,0.44,0.13}{\textbf{{#1}}}}
    \newcommand{\OperatorTok}[1]{\textcolor[rgb]{0.40,0.40,0.40}{{#1}}}
    \newcommand{\BuiltInTok}[1]{{#1}}
    \newcommand{\ExtensionTok}[1]{{#1}}
    \newcommand{\PreprocessorTok}[1]{\textcolor[rgb]{0.74,0.48,0.00}{{#1}}}
    \newcommand{\AttributeTok}[1]{\textcolor[rgb]{0.49,0.56,0.16}{{#1}}}
    \newcommand{\InformationTok}[1]{\textcolor[rgb]{0.38,0.63,0.69}{\textbf{\textit{{#1}}}}}
    \newcommand{\WarningTok}[1]{\textcolor[rgb]{0.38,0.63,0.69}{\textbf{\textit{{#1}}}}}
    
    
    % Define a nice break command that doesn't care if a line doesn't already
    % exist.
    \def\br{\hspace*{\fill} \\* }
    % Math Jax compatability definitions
    \def\gt{>}
    \def\lt{<}
    % Document parameters
    \title{PCA\_face}
    
    
    

    % Pygments definitions
    
\makeatletter
\def\PY@reset{\let\PY@it=\relax \let\PY@bf=\relax%
    \let\PY@ul=\relax \let\PY@tc=\relax%
    \let\PY@bc=\relax \let\PY@ff=\relax}
\def\PY@tok#1{\csname PY@tok@#1\endcsname}
\def\PY@toks#1+{\ifx\relax#1\empty\else%
    \PY@tok{#1}\expandafter\PY@toks\fi}
\def\PY@do#1{\PY@bc{\PY@tc{\PY@ul{%
    \PY@it{\PY@bf{\PY@ff{#1}}}}}}}
\def\PY#1#2{\PY@reset\PY@toks#1+\relax+\PY@do{#2}}

\expandafter\def\csname PY@tok@w\endcsname{\def\PY@tc##1{\textcolor[rgb]{0.73,0.73,0.73}{##1}}}
\expandafter\def\csname PY@tok@c\endcsname{\let\PY@it=\textit\def\PY@tc##1{\textcolor[rgb]{0.25,0.50,0.50}{##1}}}
\expandafter\def\csname PY@tok@cp\endcsname{\def\PY@tc##1{\textcolor[rgb]{0.74,0.48,0.00}{##1}}}
\expandafter\def\csname PY@tok@k\endcsname{\let\PY@bf=\textbf\def\PY@tc##1{\textcolor[rgb]{0.00,0.50,0.00}{##1}}}
\expandafter\def\csname PY@tok@kp\endcsname{\def\PY@tc##1{\textcolor[rgb]{0.00,0.50,0.00}{##1}}}
\expandafter\def\csname PY@tok@kt\endcsname{\def\PY@tc##1{\textcolor[rgb]{0.69,0.00,0.25}{##1}}}
\expandafter\def\csname PY@tok@o\endcsname{\def\PY@tc##1{\textcolor[rgb]{0.40,0.40,0.40}{##1}}}
\expandafter\def\csname PY@tok@ow\endcsname{\let\PY@bf=\textbf\def\PY@tc##1{\textcolor[rgb]{0.67,0.13,1.00}{##1}}}
\expandafter\def\csname PY@tok@nb\endcsname{\def\PY@tc##1{\textcolor[rgb]{0.00,0.50,0.00}{##1}}}
\expandafter\def\csname PY@tok@nf\endcsname{\def\PY@tc##1{\textcolor[rgb]{0.00,0.00,1.00}{##1}}}
\expandafter\def\csname PY@tok@nc\endcsname{\let\PY@bf=\textbf\def\PY@tc##1{\textcolor[rgb]{0.00,0.00,1.00}{##1}}}
\expandafter\def\csname PY@tok@nn\endcsname{\let\PY@bf=\textbf\def\PY@tc##1{\textcolor[rgb]{0.00,0.00,1.00}{##1}}}
\expandafter\def\csname PY@tok@ne\endcsname{\let\PY@bf=\textbf\def\PY@tc##1{\textcolor[rgb]{0.82,0.25,0.23}{##1}}}
\expandafter\def\csname PY@tok@nv\endcsname{\def\PY@tc##1{\textcolor[rgb]{0.10,0.09,0.49}{##1}}}
\expandafter\def\csname PY@tok@no\endcsname{\def\PY@tc##1{\textcolor[rgb]{0.53,0.00,0.00}{##1}}}
\expandafter\def\csname PY@tok@nl\endcsname{\def\PY@tc##1{\textcolor[rgb]{0.63,0.63,0.00}{##1}}}
\expandafter\def\csname PY@tok@ni\endcsname{\let\PY@bf=\textbf\def\PY@tc##1{\textcolor[rgb]{0.60,0.60,0.60}{##1}}}
\expandafter\def\csname PY@tok@na\endcsname{\def\PY@tc##1{\textcolor[rgb]{0.49,0.56,0.16}{##1}}}
\expandafter\def\csname PY@tok@nt\endcsname{\let\PY@bf=\textbf\def\PY@tc##1{\textcolor[rgb]{0.00,0.50,0.00}{##1}}}
\expandafter\def\csname PY@tok@nd\endcsname{\def\PY@tc##1{\textcolor[rgb]{0.67,0.13,1.00}{##1}}}
\expandafter\def\csname PY@tok@s\endcsname{\def\PY@tc##1{\textcolor[rgb]{0.73,0.13,0.13}{##1}}}
\expandafter\def\csname PY@tok@sd\endcsname{\let\PY@it=\textit\def\PY@tc##1{\textcolor[rgb]{0.73,0.13,0.13}{##1}}}
\expandafter\def\csname PY@tok@si\endcsname{\let\PY@bf=\textbf\def\PY@tc##1{\textcolor[rgb]{0.73,0.40,0.53}{##1}}}
\expandafter\def\csname PY@tok@se\endcsname{\let\PY@bf=\textbf\def\PY@tc##1{\textcolor[rgb]{0.73,0.40,0.13}{##1}}}
\expandafter\def\csname PY@tok@sr\endcsname{\def\PY@tc##1{\textcolor[rgb]{0.73,0.40,0.53}{##1}}}
\expandafter\def\csname PY@tok@ss\endcsname{\def\PY@tc##1{\textcolor[rgb]{0.10,0.09,0.49}{##1}}}
\expandafter\def\csname PY@tok@sx\endcsname{\def\PY@tc##1{\textcolor[rgb]{0.00,0.50,0.00}{##1}}}
\expandafter\def\csname PY@tok@m\endcsname{\def\PY@tc##1{\textcolor[rgb]{0.40,0.40,0.40}{##1}}}
\expandafter\def\csname PY@tok@gh\endcsname{\let\PY@bf=\textbf\def\PY@tc##1{\textcolor[rgb]{0.00,0.00,0.50}{##1}}}
\expandafter\def\csname PY@tok@gu\endcsname{\let\PY@bf=\textbf\def\PY@tc##1{\textcolor[rgb]{0.50,0.00,0.50}{##1}}}
\expandafter\def\csname PY@tok@gd\endcsname{\def\PY@tc##1{\textcolor[rgb]{0.63,0.00,0.00}{##1}}}
\expandafter\def\csname PY@tok@gi\endcsname{\def\PY@tc##1{\textcolor[rgb]{0.00,0.63,0.00}{##1}}}
\expandafter\def\csname PY@tok@gr\endcsname{\def\PY@tc##1{\textcolor[rgb]{1.00,0.00,0.00}{##1}}}
\expandafter\def\csname PY@tok@ge\endcsname{\let\PY@it=\textit}
\expandafter\def\csname PY@tok@gs\endcsname{\let\PY@bf=\textbf}
\expandafter\def\csname PY@tok@gp\endcsname{\let\PY@bf=\textbf\def\PY@tc##1{\textcolor[rgb]{0.00,0.00,0.50}{##1}}}
\expandafter\def\csname PY@tok@go\endcsname{\def\PY@tc##1{\textcolor[rgb]{0.53,0.53,0.53}{##1}}}
\expandafter\def\csname PY@tok@gt\endcsname{\def\PY@tc##1{\textcolor[rgb]{0.00,0.27,0.87}{##1}}}
\expandafter\def\csname PY@tok@err\endcsname{\def\PY@bc##1{\setlength{\fboxsep}{0pt}\fcolorbox[rgb]{1.00,0.00,0.00}{1,1,1}{\strut ##1}}}
\expandafter\def\csname PY@tok@kc\endcsname{\let\PY@bf=\textbf\def\PY@tc##1{\textcolor[rgb]{0.00,0.50,0.00}{##1}}}
\expandafter\def\csname PY@tok@kd\endcsname{\let\PY@bf=\textbf\def\PY@tc##1{\textcolor[rgb]{0.00,0.50,0.00}{##1}}}
\expandafter\def\csname PY@tok@kn\endcsname{\let\PY@bf=\textbf\def\PY@tc##1{\textcolor[rgb]{0.00,0.50,0.00}{##1}}}
\expandafter\def\csname PY@tok@kr\endcsname{\let\PY@bf=\textbf\def\PY@tc##1{\textcolor[rgb]{0.00,0.50,0.00}{##1}}}
\expandafter\def\csname PY@tok@bp\endcsname{\def\PY@tc##1{\textcolor[rgb]{0.00,0.50,0.00}{##1}}}
\expandafter\def\csname PY@tok@fm\endcsname{\def\PY@tc##1{\textcolor[rgb]{0.00,0.00,1.00}{##1}}}
\expandafter\def\csname PY@tok@vc\endcsname{\def\PY@tc##1{\textcolor[rgb]{0.10,0.09,0.49}{##1}}}
\expandafter\def\csname PY@tok@vg\endcsname{\def\PY@tc##1{\textcolor[rgb]{0.10,0.09,0.49}{##1}}}
\expandafter\def\csname PY@tok@vi\endcsname{\def\PY@tc##1{\textcolor[rgb]{0.10,0.09,0.49}{##1}}}
\expandafter\def\csname PY@tok@vm\endcsname{\def\PY@tc##1{\textcolor[rgb]{0.10,0.09,0.49}{##1}}}
\expandafter\def\csname PY@tok@sa\endcsname{\def\PY@tc##1{\textcolor[rgb]{0.73,0.13,0.13}{##1}}}
\expandafter\def\csname PY@tok@sb\endcsname{\def\PY@tc##1{\textcolor[rgb]{0.73,0.13,0.13}{##1}}}
\expandafter\def\csname PY@tok@sc\endcsname{\def\PY@tc##1{\textcolor[rgb]{0.73,0.13,0.13}{##1}}}
\expandafter\def\csname PY@tok@dl\endcsname{\def\PY@tc##1{\textcolor[rgb]{0.73,0.13,0.13}{##1}}}
\expandafter\def\csname PY@tok@s2\endcsname{\def\PY@tc##1{\textcolor[rgb]{0.73,0.13,0.13}{##1}}}
\expandafter\def\csname PY@tok@sh\endcsname{\def\PY@tc##1{\textcolor[rgb]{0.73,0.13,0.13}{##1}}}
\expandafter\def\csname PY@tok@s1\endcsname{\def\PY@tc##1{\textcolor[rgb]{0.73,0.13,0.13}{##1}}}
\expandafter\def\csname PY@tok@mb\endcsname{\def\PY@tc##1{\textcolor[rgb]{0.40,0.40,0.40}{##1}}}
\expandafter\def\csname PY@tok@mf\endcsname{\def\PY@tc##1{\textcolor[rgb]{0.40,0.40,0.40}{##1}}}
\expandafter\def\csname PY@tok@mh\endcsname{\def\PY@tc##1{\textcolor[rgb]{0.40,0.40,0.40}{##1}}}
\expandafter\def\csname PY@tok@mi\endcsname{\def\PY@tc##1{\textcolor[rgb]{0.40,0.40,0.40}{##1}}}
\expandafter\def\csname PY@tok@il\endcsname{\def\PY@tc##1{\textcolor[rgb]{0.40,0.40,0.40}{##1}}}
\expandafter\def\csname PY@tok@mo\endcsname{\def\PY@tc##1{\textcolor[rgb]{0.40,0.40,0.40}{##1}}}
\expandafter\def\csname PY@tok@ch\endcsname{\let\PY@it=\textit\def\PY@tc##1{\textcolor[rgb]{0.25,0.50,0.50}{##1}}}
\expandafter\def\csname PY@tok@cm\endcsname{\let\PY@it=\textit\def\PY@tc##1{\textcolor[rgb]{0.25,0.50,0.50}{##1}}}
\expandafter\def\csname PY@tok@cpf\endcsname{\let\PY@it=\textit\def\PY@tc##1{\textcolor[rgb]{0.25,0.50,0.50}{##1}}}
\expandafter\def\csname PY@tok@c1\endcsname{\let\PY@it=\textit\def\PY@tc##1{\textcolor[rgb]{0.25,0.50,0.50}{##1}}}
\expandafter\def\csname PY@tok@cs\endcsname{\let\PY@it=\textit\def\PY@tc##1{\textcolor[rgb]{0.25,0.50,0.50}{##1}}}

\def\PYZbs{\char`\\}
\def\PYZus{\char`\_}
\def\PYZob{\char`\{}
\def\PYZcb{\char`\}}
\def\PYZca{\char`\^}
\def\PYZam{\char`\&}
\def\PYZlt{\char`\<}
\def\PYZgt{\char`\>}
\def\PYZsh{\char`\#}
\def\PYZpc{\char`\%}
\def\PYZdl{\char`\$}
\def\PYZhy{\char`\-}
\def\PYZsq{\char`\'}
\def\PYZdq{\char`\"}
\def\PYZti{\char`\~}
% for compatibility with earlier versions
\def\PYZat{@}
\def\PYZlb{[}
\def\PYZrb{]}
\makeatother


    % Exact colors from NB
    \definecolor{incolor}{rgb}{0.0, 0.0, 0.5}
    \definecolor{outcolor}{rgb}{0.545, 0.0, 0.0}



    
    % Prevent overflowing lines due to hard-to-break entities
    \sloppy 
    % Setup hyperref package
    \hypersetup{
      breaklinks=true,  % so long urls are correctly broken across lines
      colorlinks=true,
      urlcolor=urlcolor,
      linkcolor=linkcolor,
      citecolor=citecolor,
      }
    % Slightly bigger margins than the latex defaults
    
    \geometry{verbose,tmargin=1in,bmargin=1in,lmargin=1in,rmargin=1in}
    
    

    \begin{document}
    
    
    \maketitle
    
    

    
    \begin{Verbatim}[commandchars=\\\{\}]
{\color{incolor}In [{\color{incolor}1}]:} \PY{k+kn}{import} \PY{n+nn}{os}
        \PY{k+kn}{import} \PY{n+nn}{numpy} \PY{k}{as} \PY{n+nn}{np}
        \PY{k+kn}{import} \PY{n+nn}{pandas} \PY{k}{as} \PY{n+nn}{pd}
        \PY{k+kn}{import} \PY{n+nn}{matplotlib}\PY{n+nn}{.}\PY{n+nn}{pyplot} \PY{k}{as} \PY{n+nn}{plt}
        \PY{k+kn}{import} \PY{n+nn}{matplotlib} \PY{k}{as} \PY{n+nn}{mpl}
        \PY{k+kn}{import} \PY{n+nn}{cv2}
        \PY{k+kn}{import} \PY{n+nn}{re}
        
        \PY{n}{mpl}\PY{o}{.}\PY{n}{rcParams}\PY{p}{[}\PY{l+s+s1}{\PYZsq{}}\PY{l+s+s1}{font.sans\PYZhy{}serif}\PY{l+s+s1}{\PYZsq{}}\PY{p}{]} \PY{o}{=} \PY{p}{[}\PY{l+s+s1}{\PYZsq{}}\PY{l+s+s1}{SimHei}\PY{l+s+s1}{\PYZsq{}}\PY{p}{]}
        \PY{o}{\PYZpc{}}\PY{k}{matplotlib} inline
        
        \PY{k}{for} \PY{n}{lib} \PY{o+ow}{in} \PY{p}{[}\PY{p}{(}\PY{l+s+s1}{\PYZsq{}}\PY{l+s+s1}{numpy}\PY{l+s+s1}{\PYZsq{}}\PY{p}{,}\PY{n}{np}\PY{p}{)}\PY{p}{,} \PY{p}{(}\PY{l+s+s1}{\PYZsq{}}\PY{l+s+s1}{pandas}\PY{l+s+s1}{\PYZsq{}}\PY{p}{,} \PY{n}{pd}\PY{p}{)}\PY{p}{,} \PY{p}{(}\PY{l+s+s1}{\PYZsq{}}\PY{l+s+s1}{matplotlib}\PY{l+s+s1}{\PYZsq{}}\PY{p}{,} \PY{n}{mpl}\PY{p}{)}\PY{p}{,} \PY{p}{(}\PY{l+s+s1}{\PYZsq{}}\PY{l+s+s1}{opencv}\PY{l+s+s1}{\PYZsq{}}\PY{p}{,} \PY{n}{cv2}\PY{p}{)}\PY{p}{]}\PY{p}{:}
            \PY{n+nb}{print}\PY{p}{(}\PY{n}{f}\PY{l+s+s1}{\PYZsq{}}\PY{l+s+si}{\PYZob{}lib[0]:\PYZgt{}10\PYZcb{}}\PY{l+s+s1}{ version is }\PY{l+s+si}{\PYZob{}lib[1].\PYZus{}\PYZus{}version\PYZus{}\PYZus{}\PYZcb{}}\PY{l+s+s1}{\PYZsq{}}\PY{p}{)}
\end{Verbatim}


    \begin{Verbatim}[commandchars=\\\{\}]
     numpy version is 1.17.2
    pandas version is 0.25.1
matplotlib version is 3.1.1
    opencv version is 3.4.1

    \end{Verbatim}

    \begin{Verbatim}[commandchars=\\\{\}]
{\color{incolor}In [{\color{incolor}2}]:} \PY{k}{def} \PY{n+nf}{test\PYZus{}train\PYZus{}file\PYZus{}names}\PY{p}{(}\PY{n}{k}\PY{p}{,} \PY{n}{d}\PY{p}{)}\PY{p}{:}
            \PY{l+s+sd}{\PYZdq{}\PYZdq{}\PYZdq{}}
        \PY{l+s+sd}{    将指定目录下的文件分组,每组随机提取 k 比率的文件作为训练集,其余的作为测试集。}
        \PY{l+s+sd}{    注意:该函数只对文件名操作,并非实际对文件操作}
        \PY{l+s+sd}{    }
        \PY{l+s+sd}{    Paramaters}
        \PY{l+s+sd}{    \PYZhy{}\PYZhy{}\PYZhy{}\PYZhy{}\PYZhy{}\PYZhy{}\PYZhy{}\PYZhy{}\PYZhy{}\PYZhy{}\PYZhy{}}
        \PY{l+s+sd}{    k \PYZhy{}\PYZhy{} 训练集所占比率 0\PYZlt{}k\PYZlt{}=1}
        \PY{l+s+sd}{    d \PYZhy{}\PYZhy{} 存放文件的目录}
        \PY{l+s+sd}{    }
        \PY{l+s+sd}{    Returns}
        \PY{l+s+sd}{    \PYZhy{}\PYZhy{}\PYZhy{}\PYZhy{}\PYZhy{}\PYZhy{}\PYZhy{}}
        \PY{l+s+sd}{    ndarray 1\PYZhy{}D 训练集}
        \PY{l+s+sd}{    ndarray 1\PYZhy{}D 测试集}
        \PY{l+s+sd}{    \PYZdq{}\PYZdq{}\PYZdq{}}
            \PY{c+c1}{\PYZsh{} 构建的DataFrame的字段名称}
            \PY{n}{FILE\PYZus{}NAME} \PY{o}{=} \PY{l+s+s1}{\PYZsq{}}\PY{l+s+s1}{file\PYZus{}name}\PY{l+s+s1}{\PYZsq{}}
            \PY{n}{FILE\PYZus{}GROUP\PYZus{}NAME} \PY{o}{=} \PY{l+s+s1}{\PYZsq{}}\PY{l+s+s1}{file\PYZus{}group\PYZus{}name}\PY{l+s+s1}{\PYZsq{}}
            \PY{c+c1}{\PYZsh{} 获取文件列表}
            \PY{n}{fn\PYZus{}list} \PY{o}{=} \PY{n}{os}\PY{o}{.}\PY{n}{listdir}\PY{p}{(}\PY{n}{d}\PY{p}{)}
            \PY{c+c1}{\PYZsh{} 根据文件列表创建 DataFrame}
            \PY{n}{df} \PY{o}{=} \PY{n}{pd}\PY{o}{.}\PY{n}{DataFrame}\PY{p}{(}\PY{n}{fn\PYZus{}list}\PY{p}{,} \PY{n}{columns}\PY{o}{=}\PY{p}{[}\PY{n}{FILE\PYZus{}NAME}\PY{p}{]}\PY{p}{)}
            \PY{c+c1}{\PYZsh{} 增加文件文组标示}
            \PY{c+c1}{\PYZsh{} extract 的说明: 根据正则表达式 r\PYZsq{}(.*)(\PYZus{})\PYZsq{} 提取字符串}
            \PY{c+c1}{\PYZsh{}                比如: \PYZsq{}s1\PYZus{}1.bmp\PYZsq{} \PYZhy{}\PYZgt{} (\PYZsq{}s1\PYZsq{}, \PYZsq{}\PYZus{}\PYZsq{})}
            \PY{c+c1}{\PYZsh{}                     \PYZsq{}s10\PYZus{}10.bmp\PYZsq{} \PYZhy{}\PYZgt{} (\PYZsq{}s10\PYZsq{}, \PYZsq{}\PYZus{}\PYZsq{})}
            \PY{n}{df}\PY{p}{[}\PY{n}{FILE\PYZus{}GROUP\PYZus{}NAME}\PY{p}{]} \PY{o}{=} \PY{n}{df}\PY{o}{.}\PY{n}{file\PYZus{}name}\PY{o}{.}\PY{n}{str}\PY{o}{.}\PY{n}{extract}\PY{p}{(}\PY{l+s+sa}{r}\PY{l+s+s1}{\PYZsq{}}\PY{l+s+s1}{(.*)(\PYZus{})}\PY{l+s+s1}{\PYZsq{}}\PY{p}{)}\PY{p}{[}\PY{l+m+mi}{0}\PY{p}{]}
            \PY{c+c1}{\PYZsh{} 根据 FILE\PYZus{}GROUP\PYZus{}NAME 进行分组, 得到分组对象 grouped}
            \PY{c+c1}{\PYZsh{} 分组对象 grouded 的 groups 属性是一个字典, key 是分组名称, value 是分组数据(row index)}
            \PY{c+c1}{\PYZsh{} 形如: \PYZob{}\PYZsq{}s1\PYZsq{}: [1,2,3,...], \PYZsq{}s2\PYZsq{}: [10,11,12...]\PYZcb{}}
            \PY{n}{grouped} \PY{o}{=} \PY{n}{df}\PY{o}{.}\PY{n}{groupby}\PY{p}{(}\PY{p}{[}\PY{n}{FILE\PYZus{}GROUP\PYZus{}NAME}\PY{p}{]}\PY{p}{)}
            \PY{n}{train\PYZus{}set} \PY{o}{=} \PY{n}{np}\PY{o}{.}\PY{n}{array}\PY{p}{(}\PY{p}{[}\PY{p}{]}\PY{p}{)}
            \PY{n}{test\PYZus{}set} \PY{o}{=} \PY{n}{np}\PY{o}{.}\PY{n}{array}\PY{p}{(}\PY{p}{[}\PY{p}{]}\PY{p}{)}
            \PY{c+c1}{\PYZsh{} 循环所有分组数据, 每次操作一组数据}
            \PY{k}{for} \PY{n}{v} \PY{o+ow}{in} \PY{n}{grouped}\PY{o}{.}\PY{n}{groups}\PY{o}{.}\PY{n}{values}\PY{p}{(}\PY{p}{)}\PY{p}{:}
                \PY{c+c1}{\PYZsh{} 从 DF 中获取 file\PYZus{}name 字段 的数据}
                \PY{c+c1}{\PYZsh{} v 是 row index.}
                \PY{n}{data} \PY{o}{=} \PY{n}{df}\PY{p}{[}\PY{n}{FILE\PYZus{}NAME}\PY{p}{]}\PY{p}{[}\PY{n}{v}\PY{p}{]}
                \PY{c+c1}{\PYZsh{} 随机获取指定比率的数据 作为训练集(文件名称)}
                \PY{n}{train} \PY{o}{=} \PY{n}{data}\PY{o}{.}\PY{n}{sample}\PY{p}{(}\PY{n}{frac}\PY{o}{=}\PY{n}{k}\PY{p}{)}
                \PY{c+c1}{\PYZsh{} 提取剩余数据 作为测试集(文件名称)}
                \PY{n}{test} \PY{o}{=} \PY{n}{data}\PY{o}{.}\PY{n}{drop}\PY{p}{(}\PY{n}{train}\PY{o}{.}\PY{n}{index}\PY{p}{)}
                \PY{c+c1}{\PYZsh{} 将一组中的随机提取的训练集数据放入总训练集中}
                \PY{n}{train\PYZus{}set} \PY{o}{=} \PY{n}{np}\PY{o}{.}\PY{n}{concatenate}\PY{p}{(}\PY{p}{(}\PY{n}{train\PYZus{}set}\PY{p}{,} \PY{n}{train}\PY{o}{.}\PY{n}{values}\PY{p}{)}\PY{p}{)}
                \PY{c+c1}{\PYZsh{} 将一组中的测试集数据放入总测试集中}
                \PY{n}{test\PYZus{}set} \PY{o}{=} \PY{n}{np}\PY{o}{.}\PY{n}{concatenate}\PY{p}{(}\PY{p}{(}\PY{n}{test\PYZus{}set}\PY{p}{,} \PY{n}{test}\PY{o}{.}\PY{n}{values}\PY{p}{)}\PY{p}{)}
            \PY{k}{return} \PY{n}{train\PYZus{}set}\PY{p}{,} \PY{n}{test\PYZus{}set}
        
        \PY{k}{def} \PY{n+nf}{img2vector}\PY{p}{(}\PY{n}{img\PYZus{}file\PYZus{}path\PYZus{}name}\PY{p}{)}\PY{p}{:}
            \PY{l+s+sd}{\PYZdq{}\PYZdq{}\PYZdq{}}
        \PY{l+s+sd}{    图片向量化}
        \PY{l+s+sd}{    将给定的图片(m, n), 输出(1, m*n)的行向量}
        \PY{l+s+sd}{    \PYZdq{}\PYZdq{}\PYZdq{}}
            \PY{n}{img} \PY{o}{=} \PY{n}{cv2}\PY{o}{.}\PY{n}{imread}\PY{p}{(}\PY{n}{img\PYZus{}file\PYZus{}path\PYZus{}name}\PY{p}{,} \PY{l+m+mi}{0}\PY{p}{)}
            \PY{k}{return} \PY{n}{img}\PY{o}{.}\PY{n}{flatten}\PY{p}{(}\PY{p}{)}\PY{o}{.}\PY{n}{reshape}\PY{p}{(}\PY{l+m+mi}{1}\PY{p}{,} \PY{o}{\PYZhy{}}\PY{l+m+mi}{1}\PY{p}{)}
        
        \PY{k}{def} \PY{n+nf}{load\PYZus{}orl}\PY{p}{(}\PY{n}{k}\PY{p}{,} \PY{n}{d}\PY{p}{)}\PY{p}{:}
            \PY{l+s+sd}{\PYZdq{}\PYZdq{}\PYZdq{}}
        \PY{l+s+sd}{    加载图片文件形成 matrix}
        \PY{l+s+sd}{    }
        \PY{l+s+sd}{    Parameters}
        \PY{l+s+sd}{    \PYZhy{}\PYZhy{}\PYZhy{}\PYZhy{}\PYZhy{}\PYZhy{}\PYZhy{}\PYZhy{}\PYZhy{}\PYZhy{}}
        \PY{l+s+sd}{    k \PYZob{}float\PYZcb{} \PYZhy{}\PYZhy{} 图片文件训练集比率 0 \PYZlt{} k \PYZlt{}= 1}
        \PY{l+s+sd}{    d \PYZob{}string\PYZcb{} \PYZhy{}\PYZhy{} 图片文件所在目录}
        \PY{l+s+sd}{    }
        \PY{l+s+sd}{    Returns}
        \PY{l+s+sd}{    \PYZhy{}\PYZhy{}\PYZhy{}\PYZhy{}\PYZhy{}\PYZhy{}\PYZhy{}}
        \PY{l+s+sd}{    \PYZob{}ndarray 2D\PYZcb{} \PYZhy{}\PYZhy{} 训练集(特征)}
        \PY{l+s+sd}{    \PYZob{}ndarray 2D\PYZcb{} \PYZhy{}\PYZhy{} 训练集(标签/真实值)}
        \PY{l+s+sd}{    \PYZob{}ndarray 2D\PYZcb{} \PYZhy{}\PYZhy{} 测试集(特征)}
        \PY{l+s+sd}{    \PYZob{}ndarray 2D\PYZcb{} \PYZhy{}\PYZhy{} 测试集(标签/真实值)}
        \PY{l+s+sd}{    \PYZdq{}\PYZdq{}\PYZdq{}}
            \PY{c+c1}{\PYZsh{} 图片文件拆分为 训练集用文件 测试集用文件}
            \PY{n}{train\PYZus{}file\PYZus{}names}\PY{p}{,} \PY{n}{test\PYZus{}file\PYZus{}names} \PY{o}{=} \PY{n}{test\PYZus{}train\PYZus{}file\PYZus{}names}\PY{p}{(}\PY{n}{k}\PY{p}{,} \PY{n}{d}\PY{p}{)}
            \PY{c+c1}{\PYZsh{} 生成测试集第一条数据}
            \PY{n}{train\PYZus{}set} \PY{o}{=} \PY{n}{img2vector}\PY{p}{(}\PY{n}{os}\PY{o}{.}\PY{n}{path}\PY{o}{.}\PY{n}{join}\PY{p}{(}\PY{n}{d}\PY{p}{,} \PY{n}{train\PYZus{}file\PYZus{}names}\PY{p}{[}\PY{l+m+mi}{0}\PY{p}{]}\PY{p}{)}\PY{p}{)}
            \PY{c+c1}{\PYZsh{} 生成测试集第一个标签}
            \PY{n}{train\PYZus{}label} \PY{o}{=} \PY{n}{np}\PY{o}{.}\PY{n}{array}\PY{p}{(}\PY{p}{[}\PY{n}{re}\PY{o}{.}\PY{n}{sub}\PY{p}{(}\PY{l+s+s1}{\PYZsq{}}\PY{l+s+s1}{\PYZus{}.*}\PY{l+s+s1}{\PYZsq{}}\PY{p}{,} \PY{l+s+s1}{\PYZsq{}}\PY{l+s+s1}{\PYZsq{}}\PY{p}{,} \PY{n}{train\PYZus{}file\PYZus{}names}\PY{p}{[}\PY{l+m+mi}{0}\PY{p}{]}\PY{p}{)}\PY{p}{]}\PY{p}{)}
            \PY{c+c1}{\PYZsh{} 生成训练集第一条数据}
            \PY{n}{test\PYZus{}set} \PY{o}{=} \PY{n}{img2vector}\PY{p}{(}\PY{n}{os}\PY{o}{.}\PY{n}{path}\PY{o}{.}\PY{n}{join}\PY{p}{(}\PY{n}{d}\PY{p}{,} \PY{n}{test\PYZus{}file\PYZus{}names}\PY{p}{[}\PY{l+m+mi}{0}\PY{p}{]}\PY{p}{)}\PY{p}{)}
            \PY{c+c1}{\PYZsh{} 生成训练集第一个标签}
            \PY{n}{test\PYZus{}label} \PY{o}{=} \PY{n}{np}\PY{o}{.}\PY{n}{array}\PY{p}{(}\PY{p}{[}\PY{n}{re}\PY{o}{.}\PY{n}{sub}\PY{p}{(}\PY{l+s+s1}{\PYZsq{}}\PY{l+s+s1}{\PYZus{}.*}\PY{l+s+s1}{\PYZsq{}}\PY{p}{,} \PY{l+s+s1}{\PYZsq{}}\PY{l+s+s1}{\PYZsq{}}\PY{p}{,} \PY{n}{test\PYZus{}file\PYZus{}names}\PY{p}{[}\PY{l+m+mi}{0}\PY{p}{]}\PY{p}{)}\PY{p}{]}\PY{p}{)}
            \PY{c+c1}{\PYZsh{} 循环生成训练集数据和训练集标签}
            \PY{c+c1}{\PYZsh{} 因为第一条数据已经生成过, 因此需要从第二条开始循环}
            \PY{k}{for} \PY{n}{f} \PY{o+ow}{in} \PY{n}{train\PYZus{}file\PYZus{}names}\PY{p}{[}\PY{l+m+mi}{1}\PY{p}{:}\PY{p}{]}\PY{p}{:}
                \PY{n}{img} \PY{o}{=} \PY{n}{img2vector}\PY{p}{(}\PY{n}{os}\PY{o}{.}\PY{n}{path}\PY{o}{.}\PY{n}{join}\PY{p}{(}\PY{n}{d}\PY{p}{,} \PY{n}{f}\PY{p}{)}\PY{p}{)}
                \PY{n}{train\PYZus{}set} \PY{o}{=} \PY{n}{np}\PY{o}{.}\PY{n}{concatenate}\PY{p}{(}\PY{p}{(}\PY{n}{train\PYZus{}set}\PY{p}{,} \PY{n}{img}\PY{p}{)}\PY{p}{)}
                \PY{n}{train\PYZus{}label} \PY{o}{=} \PY{n}{np}\PY{o}{.}\PY{n}{append}\PY{p}{(}\PY{n}{train\PYZus{}label}\PY{p}{,} \PY{n}{re}\PY{o}{.}\PY{n}{sub}\PY{p}{(}\PY{l+s+s1}{\PYZsq{}}\PY{l+s+s1}{\PYZus{}.*}\PY{l+s+s1}{\PYZsq{}}\PY{p}{,} \PY{l+s+s1}{\PYZsq{}}\PY{l+s+s1}{\PYZsq{}}\PY{p}{,} \PY{n}{f}\PY{p}{)}\PY{p}{)}
            \PY{c+c1}{\PYZsh{} 循环生成测试集数据和训练集标签}
            \PY{c+c1}{\PYZsh{} 因为第一条数据已经生成过, 因此需要从第二条开始循环}
            \PY{k}{for} \PY{n}{f} \PY{o+ow}{in} \PY{n}{test\PYZus{}file\PYZus{}names}\PY{p}{[}\PY{l+m+mi}{1}\PY{p}{:}\PY{p}{]}\PY{p}{:}
                \PY{n}{img} \PY{o}{=} \PY{n}{img2vector}\PY{p}{(}\PY{n}{os}\PY{o}{.}\PY{n}{path}\PY{o}{.}\PY{n}{join}\PY{p}{(}\PY{n}{d}\PY{p}{,} \PY{n}{f}\PY{p}{)}\PY{p}{)}
                \PY{n}{test\PYZus{}set} \PY{o}{=} \PY{n}{np}\PY{o}{.}\PY{n}{concatenate}\PY{p}{(}\PY{p}{(}\PY{n}{test\PYZus{}set}\PY{p}{,} \PY{n}{img}\PY{p}{)}\PY{p}{)}
                \PY{n}{test\PYZus{}label} \PY{o}{=} \PY{n}{np}\PY{o}{.}\PY{n}{append}\PY{p}{(}\PY{n}{test\PYZus{}label}\PY{p}{,} \PY{n}{re}\PY{o}{.}\PY{n}{sub}\PY{p}{(}\PY{l+s+s1}{\PYZsq{}}\PY{l+s+s1}{\PYZus{}.*}\PY{l+s+s1}{\PYZsq{}}\PY{p}{,} \PY{l+s+s1}{\PYZsq{}}\PY{l+s+s1}{\PYZsq{}}\PY{p}{,} \PY{n}{f}\PY{p}{)}\PY{p}{)}
            \PY{k}{return} \PY{n}{train\PYZus{}set}\PY{p}{,} \PY{n}{train\PYZus{}label}\PY{p}{,} \PY{n}{test\PYZus{}set}\PY{p}{,} \PY{n}{test\PYZus{}label}
        
        \PY{k}{def} \PY{n+nf}{PCA}\PY{p}{(}\PY{n}{data}\PY{p}{,} \PY{n}{r}\PY{p}{)}\PY{p}{:}
            \PY{l+s+sd}{\PYZdq{}\PYZdq{}\PYZdq{}}
        \PY{l+s+sd}{    Principle Component Analysis (PCA) 主成分分析 算法}
        
        \PY{l+s+sd}{    Parameters}
        \PY{l+s+sd}{    \PYZhy{}\PYZhy{}\PYZhy{}\PYZhy{}\PYZhy{}\PYZhy{}\PYZhy{}\PYZhy{}\PYZhy{}}
        \PY{l+s+sd}{    data \PYZob{}ndarray 2D\PYZcb{} \PYZhy{}\PYZhy{} 原数据}
        \PY{l+s+sd}{    r \PYZob{}int\PYZcb{} \PYZhy{}\PYZhy{} 降维的目标维度}
        
        \PY{l+s+sd}{    Returns}
        \PY{l+s+sd}{    \PYZhy{}\PYZhy{}\PYZhy{}\PYZhy{}\PYZhy{}\PYZhy{}\PYZhy{}}
        \PY{l+s+sd}{    \PYZob{}ndarray 2D\PYZcb{} \PYZhy{}\PYZhy{} PCA结果}
        \PY{l+s+sd}{    \PYZob{}ndarray 2D\PYZcb{} \PYZhy{}\PYZhy{} 协方差矩阵}
        \PY{l+s+sd}{    \PYZob{}ndarray 1D\PYZcb{} \PYZhy{}\PYZhy{} 协方差矩阵特征值}
        \PY{l+s+sd}{    \PYZob{}ndarray 2D\PYZcb{} \PYZhy{}\PYZhy{} 协方差矩阵特征向量}
        \PY{l+s+sd}{    \PYZdq{}\PYZdq{}\PYZdq{}}
            \PY{c+c1}{\PYZsh{} 原始数据的记录量,即行的数量}
            \PY{n}{rows} \PY{o}{=} \PY{n}{data}\PY{o}{.}\PY{n}{shape}\PY{p}{[}\PY{l+m+mi}{0}\PY{p}{]}
            \PY{c+c1}{\PYZsh{} 复制一份原数据并进行转置}
            \PY{c+c1}{\PYZsh{} 转置后的矩阵每行代表一个特征,每列代表一条记录(一张图片)}
            \PY{n}{X} \PY{o}{=} \PY{n}{np}\PY{o}{.}\PY{n}{copy}\PY{p}{(}\PY{n}{data}\PY{p}{)}\PY{o}{.}\PY{n}{T}
            \PY{c+c1}{\PYZsh{} 按行(特征)计算特征的平均值}
            \PY{n}{mu} \PY{o}{=} \PY{n}{X}\PY{o}{.}\PY{n}{mean}\PY{p}{(}\PY{n}{axis}\PY{o}{=}\PY{l+m+mi}{1}\PY{p}{)}\PY{p}{[}\PY{p}{:}\PY{p}{,} \PY{k+kc}{None}\PY{p}{]}
            \PY{c+c1}{\PYZsh{} 按行(特征)进行均值归零化}
            \PY{n}{A} \PY{o}{=} \PY{n}{X} \PY{o}{\PYZhy{}} \PY{n}{mu}
            \PY{c+c1}{\PYZsh{} 计算协方差}
            \PY{n}{C} \PY{o}{=} \PY{p}{(}\PY{n}{A} \PY{o}{@} \PY{n}{A}\PY{o}{.}\PY{n}{T}\PY{p}{)} \PY{o}{/} \PY{n}{rows}
            \PY{c+c1}{\PYZsh{} 求协方差的特征值和特征向量}
            \PY{c+c1}{\PYZsh{} 因为 eig 方法已经将特征向量进行了单位向量变换,因此无需再对特征向量进行正则化(L2)}
            \PY{n}{D}\PY{p}{,} \PY{n}{V} \PY{o}{=} \PY{n}{np}\PY{o}{.}\PY{n}{linalg}\PY{o}{.}\PY{n}{eig}\PY{p}{(}\PY{n}{C}\PY{p}{)}
            \PY{c+c1}{\PYZsh{} 获取特征值从大到小的索引}
            \PY{n}{idx} \PY{o}{=} \PY{n}{D}\PY{o}{.}\PY{n}{argsort}\PY{p}{(}\PY{p}{)}\PY{p}{[}\PY{p}{:}\PY{p}{:}\PY{o}{\PYZhy{}}\PY{l+m+mi}{1}\PY{p}{]}
            \PY{c+c1}{\PYZsh{} 将特征值从大到小排序}
            \PY{n}{D} \PY{o}{=} \PY{n}{D}\PY{p}{[}\PY{n}{idx}\PY{p}{]}
            \PY{c+c1}{\PYZsh{} 将特征向量从大到小排序}
            \PY{n}{V} \PY{o}{=} \PY{n}{V}\PY{p}{[}\PY{p}{:}\PY{p}{,} \PY{n}{idx}\PY{p}{]}
            \PY{c+c1}{\PYZsh{} 能够使协方差矩阵对角化的矩阵}
            \PY{n}{P} \PY{o}{=} \PY{n}{V}\PY{o}{.}\PY{n}{T}
            \PY{c+c1}{\PYZsh{} 用于降维的矩阵}
            \PY{n}{P\PYZus{}r} \PY{o}{=} \PY{n}{P}\PY{p}{[}\PY{l+m+mi}{0}\PY{p}{:}\PY{n}{r}\PY{p}{,} \PY{p}{:}\PY{p}{]}
            \PY{k}{return} \PY{n}{P\PYZus{}r}\PY{p}{,} \PY{n}{C}\PY{p}{,} \PY{n}{D}\PY{p}{,} \PY{n}{V}
\end{Verbatim}



    % Add a bibliography block to the postdoc
    
    
    
    \end{document}
